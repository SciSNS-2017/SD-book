%
% \section{Lezione del 23/10/18 [Bindini]}
%

\section{Prodotto scalare e struttura topologica}
\emph{Setting}: $ V $ è uno spazio vettoriale su $ \K = \R \text{ o } \C $. 

\begin{definition}[prodotto scalare]
    Una funzione $ {\langle \; , \, \rangle} \colon V \times V \to \K $ è un prodotto scalare se 
    \begin{enumerate}[label = (\roman*)]
        \item $ \forall v \in V, \ {\langle v, v \rangle} \geq 0 $ e $ {\langle v, v \rangle} \iff v = 0 $;
        \item $ \forall a_1, a_2 \in \K, \forall v_1, v_2, w \in V, \ {\langle a_1v_1 + a_2v_2, w \rangle} = a_1{\langle v_1, w\rangle} + a_2{\langle v_2, w\rangle} $;
        \item $ \forall b_1, b_2 \in \K, \forall w_1, w_2, v \in V, \ {\langle v, b_1w_1 + b_2w_2 \rangle} = \bar{b}_1{\langle v, w_1\rangle} + \bar{b}_2{\langle v, w_2\rangle} $
    \end{enumerate}
\end{definition}

Un prodotto scalare su $ V $ induce una funzione $ \norm{\;} \colon V \to \K  $ detta \emph{norma} definita come $ {\norm{v} \coloneqq \sqrt{{\langle v, v \rangle}}} $. Una norma a sua volta indice una distanza $ d \colon V \times V \to \K $ data da \linebreak $ d(v, w) \coloneqq \norm{v - w} $ che definisce una struttura di spazio metrico e di topologia.

\begin{proposition}[Cauchy-Schwarz]
    Per ogni $ v, w \in V $ vale $ \abs{{\langle v, w \rangle}} \leq \norm{v} \cdot \norm{w} $. 
\end{proposition}

\begin{definition}[sottospazio ortogonale]
    Se $ W \subseteq V $ è un sottospazio definiamo l'ortogonale di $ W $ come 
    \[
        W^{\perp} \coloneqq \{v \in V : \forall w \in W, \ {\langle v, w \rangle} = 0\}.
    \]
\end{definition}

\begin{lemma}
    Il prodotto scalare è una funzione continua rispetto alla topologia indotta dalla norma (e a quella euclidea in arrivo).
\end{lemma}
\begin{proof}
    Fissati $ (v_0, w_0) \in V \times V $, usando la triangolare e Cauchy-Schwarz
    \[
        \abs{{\langle v, w \rangle} - {\langle v_0, w_0 \rangle}} = \abs{{\langle v - v_0, w\rangle} - {\langle v_0, w - w_0 \rangle}} \leq \norm{v - v_0} \, \norm{w} + \norm{v_0} \, \norm{w - w_0}.
    \]
    Quindi se $ v \in B_\delta(v_0) $ e $ w \in B_\delta(w_0) $ si ha
    \[
        \abs{{\langle v, w \rangle} - {\langle v_0, w_0 \rangle}} \leq \delta(\norm{w} + \norm{w_0}) \leq \delta(\norm{w_0} + \norm{v_0} + \delta). \qedhere
    \]
\end{proof}

\begin{proposition}
    $ W^\perp $ è un sottospazio chiuso di $ V $.
\end{proposition}
\begin{proof}
    Il fatto che sia un sottospazio è ovvio per la linearità del prodotto scalare nella prima componente. Fissiamo ora $ w \in W $. Allora $ \{w\}^\perp = \{v \in V : {\langle v, w \rangle} = 0\} $ è un chiuso (controimmagine continua di un chiuso) da cui $ W^\perp = \bigcap_{w \in W} \{w\}^\perp $ è chiuso essendo intersezione di chiusi. 
\end{proof}

\begin{proposition}
    Vale che $ (W^\perp)^\perp = \overline{W} $.
\end{proposition}
\begin{proof}
    Se $ w \in W $ allora $ {\langle v, w \rangle} = 0 $ per ogni $ v \in W^\perp $ da cui $ W \subseteq (W^\perp)^\perp \Rightarrow \overline{W} \subseteq (W^\perp)^\perp $ essendo l'ortogonale di $ W^\perp $ un chiuso. \textcolor{red}{Viceversa}
\end{proof}

\section{Basi, completezza e spazi di Hilbert}

\begin{definition}[base ortonormale]
    Una base ortonormale di $ V $ è un insieme $ \{e_j\}_{j \in J} $ con $ J $ al più numerabile tale che
    \begin{enumerate}[label=(\roman*)]
        \item $ \forall j, k \in J, \ {\langle e_j, e_k \rangle} = \delta_{jk} $;
        \item\label{def:coin:base} $ \forall v \in V, \ \exists \{v_j\}_{j \in J} \subseteq \K : v = \displaystyle{\sum_{j \in J} v_j e_j} $. 
    \end{enumerate}
\end{definition} 

Nel caso in cui $ J $ sia numerabile la condizione \ref{def:coin:base} è da intendersi come convergenza rispetto alla norma, cioè
\[
    \norm{v - \sum_{j = 1}^{N} v_j e_j} \to 0 \quad \text{ per } N \to +\infty
\]

\begin{proposition}
   Uno spazio vettoriale $ V $ è separabile\footnote{Uno spazio topologico si dice separabile se contiene un sottoinsieme denso numerabile.} se e solo se ammette una base al più numerabile.
\end{proposition}

\begin{definition}[spazio di Hilbert]
    $ V $ si dice spazio di Hilbert se è separabile e completo rispetto alla norma indotta dal prodotto scalare.
\end{definition}

\begin{thm}[criterio di completezza]
    Sia $ (V, \norm{\;}) $ uno spazio normato. Allora le seguenti proprietà sono equivalenti.
    \begin{enumerate}[label=(\roman*)]
        \item $ V $ è completo.
        \item $ \forall (v_k)_{k \in \N} \subseteq V, \ \displaystyle{\sum_{k \in \N} \norm{v_k} < +\infty \Rightarrow \sum_{k \in \N} v_k} \text{ converge} $. 
    \end{enumerate}
\end{thm}

\begin{example}[$ \R^n, \C^n $]
    Lo spazio $ \C^n $ ($ \R^n $) con il prodotto hermitiano standard (scalare standard) $ {\langle v, w \rangle} = v \cdot w \coloneqq v_1\bar{w}_1 + \ldots v_n\bar{w}_n $ è uno spazio di Hilbert. Più in generale su $ \R^n $ possiamo definire il prodotto scalare $ {\langle v, w \rangle} = v \cdot Qw $ con $ Q $ matrice simmetrica definita positiva che dota $ \R^n $ di una struttura di spazio di Hilbert, di cui una base ortonormale è data dagli autovettori di $ Q $.  
\end{example}

\begin{example}[spazio $ L^2 $]
    Consideriamo lo spazio $ L^2(X, \mathcal{F}, \mu) \coloneqq \faktor{\mathscr{L}^2(X,\mathcal{F},\mu)}{\sim} $ dove
    \[
        \mathscr{L}^2(X,\mathcal{F},\mu) \coloneqq \left\{f \colon X \to \K \ \text{ misurabili tali che } \int_{X} \abs{f}^2 \dif{\mu} < +\infty\right\}
    \]
    e $ f \sim g \iff \mu\text{-q.o.} \ f = g $. Su $ L^2(X, \mathcal{F}, \mu) $ definiamo il prodotto scalare 
    \[
        {\langle f, g \rangle} \coloneqq \int_{X} f(x) \overline{g(x)} \dif{\mu(x)}
    \]
    che induce la norma
    \[
        \norm{f}_2 \coloneqq \left(\int_{X} \abs{f(x)}^2 \dif{\mu(x)}\right)^{1/2}.
    \]
    $ L^2(X, \mathcal{F}, \mu) $ con questo prodotto scalare è uno spazio di Hilbert. 
\end{example}

\begin{example}[spazio $ l^2 $]
    Consideriamo lo spazio
    \[
        \ell^2(\N) \coloneqq \left\{(x_n)_{n \in \N} \subseteq \K : \sum_{n \in \N} \abs{x_n}^2 < +\infty\right\}.
    \]
    Tale spazio può essere visto come un esempio di $ \mathscr{L}^2(X, \mathcal{F}, \mu) $ in cui $ X = \N $, $ \mathcal{F} = \mathscr{P}(\N) $ e $ \mu(E) = \card{E} $ (misura conta punti). Infatti in tale caso le funzioni $ f \colon \N \to \K $ sono successioni a valori nel campo e l'integrale si riduce a una sommatoria. Osserviamo inoltre che, poiché l'unico insieme di misura nulla è il vuoto, la relazione di equivalenza è banale, cioè le classi di equivalenza sono i singoletti di $ \ell^2(\N) $: $ l^2(\N) \simeq \ell^2(\N) = \mathscr{L}^2(\N, \mathscr{P}(\N), \mu) $. Il prodotto scalare su $ l^2(\N) $ è
    \[
        {\langle (x_n), (y_n) \rangle} = \sum_{n \in \N} x_n \bar{y}_n.
    \]
    Tale prodotto scalare è ben definito essendo la serie assolutamente convergente (infatti $ \abs{x_n \bar{y}_n} \leq \abs{x_n}^2 /2 + \abs{y_n}^2 /2 $). Una base di $ l^2(\N) $ è $ \{e^j\}_{j \in \N} $ dove $ e^j_n \coloneqq \delta_{jn} $.
\end{example}

\begin{proposition}
    Lo spazio $ l^2(\N) $ è completo. 
\end{proposition}
\begin{proof}
    \textcolor{red}{Pezzi di 2 dimostrazioni nel \texttt{tex}.}
    \iffalse
    Sia $ (x_k)_{k \in \N} \coloneqq (x_k(n))_{k \in \N} \subseteq l^2(\N) $ una successione di Cauchy. Osserviamo che per ogni $ n \in \N $ e per ogni $ k_1, k_2 \in \N $ vale
    \[
        \abs{x_{k_1}(n) - x_{k_2}(n)}^2 \leq \sum_{n \in \N} \abs{x_{k_1}(n) - x_{k_2}(n)}^2 = \norm{x_{k_1} - x_{k_2}}_2^2.
    \]
    da cui $ \abs{x_{k_1}(n) - x_{k_2}(n)} \leq \norm{x_{k_1} - x_{k_2}}_2 $. Quindi per ogni $ n \in \N $, $ (x_k(n))_{k \in \N} \subseteq \K $ è di Cauchy e pertanto converge. Sia $ \tilde{x}(n) $ il limite e $ \tilde{x} = (\tilde{x}(n))_{n \in \N} $. \\
    Per prima cosa mostriamo che $ \tilde{x} \in l^2(\N) $. Essendo la convergenza di $ (x_k(n))_{k \in \N} $ uniforme in $ n $ si ha
    \begin{align*}
        \norm{\tilde{x}}_2^2 & = \sum_{n \in \N} \abs{\tilde{x}(n)}^2 = \sum_{n \in \N} \abs{\lim_{k \to +\infty} x_k(n)}^2 \\
        & = \sum_{n \in \N} \lim_{k \to +\infty} \abs{x_k(n)}^2 = \lim_{k \to +\infty} \sum_{n \in \N} \abs{x_k(n)}^2 \\
        & = \lim_{k \to +\infty} \norm{x_k}^2_2 = \left(\lim_{k \to +\infty} \norm{x_k}_2\right)^2.
    \end{align*}
    Ma $ \abs{\norm{x_{k_1}}_2 - \norm{x_{k_2}}_2} \leq \norm{x_{k_1} - x_{k_2}} $ quindi $ (\norm{x_k}_2)_k \subseteq \K $ è di Cauchy e quindi converge. Essendo allora $ x_k \in l^2(\N) $ per ogni $ k $ concludiamo che anche il limite di $ (\norm{x_k}_2) $ è finito da cui $ \norm{\tilde{x}}_2 < +\infty $ cioè $ \tilde{x} \in l^2(\N) $. \\
    Mostriamo ora che $ (x_k) $ converge a $ \tilde{x} $ in $ l^2(\N) $ cioè che 
    \[
        \lim_{k \to +\infty} \norm{x_k - \tilde{x}}_2^2 = \lim_{k \to +\infty} \sum_{n \in \N} \abs{x_k(n) - \tilde{x}(n)}^2 = 0.
    \]
    Osserviamo che $  $ \\


    DIM ALTERNATIVA \\
    tale che $ \sum_{k \in \N} \norm{x_k}^2 < +\infty $. Osserviamo che
    \[
        \abs{x_k(n)}^2 \leq \sum_{n \in \N} \abs{x_k(n)}^2 = \norm{x_k}^2 \quad \Rightarrow \quad \abs{x_k(n)} \leq \norm{x_k}.
    \]
    Pertanto $ \bar{x}(n) \coloneqq \sum_{k \in \N} x_k(n) $ è convergente essendo assolutamente convergente ed è il candidato limite di $ \sum_{k \in \N} x_k(n) $. Vogliamo mostrare che $ \norm{x_k - \bar{x}}_2 \to 0 $ cioè che 
    \[
        \sum_{n \in \N} \; \abs{\bar{x}(n) - \sum_{k = 1}^{m} x_k(n)} \to 0 \quad \text{ per } m \to +\infty
    \]
    \fi
\end{proof}

\begin{exercise}
    Dimostrare che lo spazio $ L^2([0, 1]) $ è completo.
\end{exercise}

%
% \section{Lezione del 30/10/18 [Bindini]}
%

\section{Convergenza debole e insiemi completi}
\emph{Setting}: $ V $ spazio di Hilbert su $ \K = \R \text{ o } \C $.

\begin{definition}[convergenza debole]
    Diciamo che $ (v_n)_{n \in \N} \subseteq V $ converge debolmente a $ v \in V $, e scriviamo $ v_n \todeb v $, se $ \forall w \in V, \ {\langle v_n, w \rangle} \to {\langle v, w \rangle} $ (o analogamente $ {\langle v_n - v, w \rangle} \to 0 $) dove la convergenza del prodotto scalare è intesa rispetto alla distanza euclidea su $ \K $. 
\end{definition}

\begin{oss}
    Su $ \C^n $ o più in generale spazi di dimensione finita la convergenza debole è equivalente a quella rispetto alla norma. Infatti per avere la convergenza in norma è sufficiente guardare la convergenza sugli elementi di una base $ \{e^j\} $ e $ v_n^j = {\langle v_n, e^j\rangle} \to {\langle v, e^j\rangle} = v^j $. Il viceversa è invece ovvio.
\end{oss}

\begin{lemma}
    Sia $ (v_n)_{n \in \N} \subseteq V $. Se $ v_n \to v $ allora $ v_n \todeb v $.
\end{lemma}

\begin{lemma}
    Sia $ (v_n)_{n \in \N} \subseteq V $. Se $ v_n \todeb v $ e $ \norm{v_n} \to \norm{v} $ allora $ v_n \to v $.
\end{lemma}
\begin{proof}
    Infatti $ \norm{v_n - v}^2 = \norm{v_n}^2 + \norm{v}^2 - 2 {\langle v_n, v \rangle} \to 2\norm{v}^2 - 2\norm{v}^2 = 0 $.
\end{proof}

\begin{thm} \label{thm:Banach-Alaoglu-facile}
    La palla $ B_1(0) \subseteq V $ è relativamente compatta rispetto alla convergenza debole, cioè per ogni $ (v_n)_{n \in \N} \subseteq V $ con $ \norm{v_n} \leq 1 $ esiste una sottosuccessione $ (v_{n_k})_{k \in \N} \subseteq V $ e un $ v \in V $ tali che $ v_{n_k} \todeb v $.
\end{thm}
\begin{proof}
    (Idea)
    Per ogni $ v \in V $, consideriamo $ D_v \coloneqq \overline{B_{\norm{v}}(0)} \subseteq \C $ e l'insieme $ K \coloneqq \prod_{v \in V} D_v $. Essendo $ D_v $ compatto per ogni $ v $, $ K $ è compatto per il Teorema di Tychonoff. Sia ora 
    \begin{align*}
        \varphi \colon B_1(0) \subseteq V & \to K \\
        w & \mapsto ({\langle w, v \rangle})_{v \in V}
    \end{align*}
    Tale mappa è ben definita essendo $ \abs{{\langle v_n, v \rangle}} \leq \norm{w} \norm{v} \leq \norm{w} $. Basterà allora mostrare che la ``restrizione di $ \varphi $ nell'immagine'' $ \psi \colon B_1(0) \to \varphi(B_1(0)) $ è continua con inversa continua (prendendo su $ K $ al topologia indotta e su $ B_1(0) $ quella indotta della convergenza debole) e che $ \varphi(B_1(0)) $. Essendo allora $ \varphi(B_1(0)) $ chiusa in un compatto otteniamo che è compatta e dato che $ \psi $ è un omeomorfismo otteniamo che $ B_1(0) $ è compatta.
\end{proof}

\begin{proposition}
    Sia $ \{e_n\}_{n \in \N} $ un insieme ortonormale in $ V $. Le seguenti proprietà sono equivalenti. 
    \begin{enumerate}[label=(\roman*)]
        \item $ \forall v \in V, \ v = \sum_{n = 0}^{+\infty} {\langle v, e_n \rangle} e_n $;
        \item $ \forall v \in V, \ \exists \{a_n\}_{n \in \N} \subseteq \K : v = \sum_{n = 0}^{+\infty} a_n e_n $;
        \item $ \operatorname{Span}{(\{e_n\})} $\footnote{Con $ \operatorname{Span}{(\{w_n\})} $ si intende l'insieme delle combinazioni lineari \emph{finite} di elementi $ w_n $.} è denso in $ V $;
        \item $ {\langle v, e_n \rangle} = 0 \ \forall n \in \N \Rightarrow v = 0 $.
    \end{enumerate}
    Se è verificata una di queste proprietà si dice che $ \{e_n\} $ è un \emph{insieme completo}.
\end{proposition}
\begin{proof}
    Mostriamo le varie implicazioni.
    \begin{description}
        \item[$ (i) \Rightarrow (ii) $] Ovvio.
        \item[$ (ii) \Rightarrow (iii) $] Prendo $ v \in V $ che scrivo come $ \sum_{n = 0}^{+\infty} a_n v_n $. Ora se $ w_{N} \coloneqq \sum_{n = 0}^{N} a_ne_n $ ho per ipotesi che $ w_N \to v $. 
        \item[$ (iii) \Rightarrow (iv) $] Sia $ v \in V $ tale che $ \forall n \in \N, \ {\langle v, e_n \rangle} = 0 $. Siano $ (w_k)_{k \in \N} \subseteq \operatorname{Span}{(\{e_n\})} $ tali che $ w_k \to v $. Ma allora $ w_k \todeb v $ quindi $ {\langle v, v \rangle} = \lim_{k} {\langle v, w_k \rangle} = 0 $ essendo $ {\langle v, w_k \rangle} = \sum_{n = 0}^{N} a_n^k {\langle v, e_n \rangle} = 0 $. Così $ \norm{v} = 0 \Rightarrow v = 0 $.
        \item[$ (iv) \Rightarrow (i) $] Fissato $ v \in V $ poniamo $ v_N \coloneqq \sum_{n = 0}^{N} {\langle v, e_n \rangle} e_n $. Essendo $ \{e_n\} $ un insieme ortonormale
        \[
        {\langle v - v_N, v_N \rangle} = {\langle v, v_N \rangle} - {\langle v_N, v_N \rangle} = \sum_{n = 0}^{N} \overline{{\langle v, e_n \rangle}}{\langle v, e_n \rangle} - \norm{v_N}^2 = 0.
        \]
        Pertanto
        \[
        \norm{v}^2 = \norm{v - v_N}^2 + \norm{v_N}^2 \geq \norm{v_N}^2 = \sum_{n = 0}^{N} \abs{{\langle v, e_n \rangle}}^2.
        \]
        Ora $ \norm{v_N} \leq \norm{v} $ quindi per il Teorema \ref{thm:Banach-Alaoglu-facile} esiste $ (v_{N_k}) $ e $ \tilde{v} $ con $ \norm{\tilde{v}} \leq \norm{v} $ tale che $ v_{N_k} \todeb \tilde{v} $. Osservando quindi che
        \[
        {\langle v - \tilde{v}, e_n \rangle} = {\langle v, e_n \rangle} - {\langle \tilde{v}, e_n \rangle} = {\langle v, e_n \rangle} - \lim_{k \to + \infty} {\langle v_{N_k}, e_n \rangle} = {\langle v, e_n \rangle} - {\langle v, e_n \rangle} = 0.
        \]
        Per ipotesi allora $ \tilde{v} = v $. Mostriamo che $ (v_{N_k}) $ tende a $ v $ in norma. Infatti
        \[
        \norm{v - v_{N_k}}^2 = {\langle v - v_{N_k}, v - v_{N_k}\rangle} = {\langle v - v_{N_k}, v \rangle} - {\langle v - v_{N_k}, v_{N_k} \rangle} = {\langle v - v_{N_k}, v \rangle} \to 0.
        \]
        Finora abbiamo dunque estratto una sottosuccessione di $ v_N $ che tende a $ v $. Possiamo applicare lo stesso procedimento su un'arbitraria sottosuccessione di $ v_N $, estraendo da ciascuna una sotto-sottosuccessione convergente a $ v $. Per un noto lemma, quindi, l'intera successione $ v_N $ deve convergere a $ v $. \qedhere
    \end{description}
\end{proof}

\section{Spazio $ L^2([0, 2\pi]) $ e serie di Fourier}
Consideriamo lo spazio $ L^2([0, 2\pi]) $.
Una sua base ortonormale di è $ e_n(\theta) = \frac{1}{\sqrt{2\pi}}e^{i n \theta} $ con $ n \in \Z $ e $ \theta \in [0, 2\pi] $ ed è detta \emph{base di Fourier}.

\begin{oss}
    Al contrario delle funzioni di $ L^2(\T^1) $, le funzioni di $ L^2([0, 2\pi]) $ possono assumere valori diversi in $ 0 $ e $ 2\pi $. Prima di passare $ [0,2\pi] $ al quoziente, quindi, i due spazi non sono del tutto analoghi.
\end{oss}

\begin{definition}[polinomi trigonometrici]
    Chiamiamo polinomi trigonometrici di grado minore o uguale a $ N $ gli elementi di
    $ T_N \coloneqq \left\{\sum_{n = -N}^{N} a_n e^{i n \theta}, \ a_n \in \C\right\} $
\end{definition}

\begin{thm}
    La base di Fourier è un insieme completo di $ L^2([0,2\pi]) $.
\end{thm}
\begin{proof}
    Road map
    \begin{enumerate}
        \item Se $ f \in L^2([0, 2\pi]) $, $ \hat{f}(n) \coloneqq {\langle f, e_n \rangle} = \frac{1}{2\pi} \int_{0}^{2\pi} f(\theta) e^{-i n\theta} \dif{\theta} $ sono i candidati coefficienti;
        \item Mostro che per ogni $ p \in T_N $ vale $ \norm{f - \sum_{n = -N}^{N} \hat{f}(n) e_n}_2 \leq \norm{f - p}_2 $;
        \item Considero la famiglia di mollificatori $ \{Q_k\} $, con $ Q_k(t) \coloneqq \lambda_k (\frac{1 + \cos{t}}{2})^k $, dove $ \lambda_k $ è il fattore di normalizzazione. Questo è un polinomio trigonometrico di grado $ k $ e che ``tende alla delta di Dirac''. Mostro che
        \[
        (f * Q_k)(x) \coloneqq \int_{-\pi}^{\pi} f(y) Q_k(x - y) \dif{y} \overset{k}{\to} f(x)
        \]
        e uso il punto precedente per concludere. \textcolor{red}{Non chiaro}.
    \end{enumerate}
\end{proof}

\begin{exercise}
    Si calcolino i coefficienti di Fourier per la funzione
    \[
    f(\theta) \coloneqq
    \begin{cases}
    1  & \text{se } 0 \leq \theta \leq \pi \\
    -1 & \text{se } \pi < \theta \leq 2\pi
    \end{cases}
    \]
\end{exercise}

\textcolor{red}{Identità di Parseval e esempi}

\section{Convergenza della serie di Fourier  \textcolor{red}{(mancante)}}
