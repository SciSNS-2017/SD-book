\documentclass[a4paper, 11pt]{scrbook}

\usepackage{style}

% Flag per escludere le immagini
\newif\iffigureon
\figureonfalse

% THEOREM ENVIORMMENT e.g. \newtheorem{comando}{nome}[numerazione]
\theoremstyle{definition}
\newtheorem{definition}{Definizione}[chapter]
\newtheorem{example}[definition]{Esempio}
\newtheorem{exercise}[definition]{Esercizio}

\theoremstyle{remark}
\newtheorem{oss}[definition]{Osservazione}

\theoremstyle{plain}
\newtheorem{thm}[definition]{Teorema}
\newtheorem{proposition}[definition]{Proposizione}
\newtheorem{lemma}[definition]{Lemma}

\newenvironment{solution}{\renewcommand\qedsymbol{}\begin{proof}[Soluzione]}{\end{proof}}

\numberwithin{equation}{chapter}

% DOCUMENT
\title{Introduzione ai sistemi dinamici I}
\author{SciSNS-2017}

\begin{document}
\frontmatter
\maketitle
Quest'opera è stata rilasciata con licenza Creative Commons Attribuzione - \linebreak Non commerciale - Condividi allo stesso modo 4.0 Internazionale. Per leggere una copia della licenza visita il sito web \url{http://creativecommons.org/licenses/by-nc-sa/4.0/}.
\iffigureon
\begin{center}
    \href{http://creativecommons.org/licenses/by-nc-sa/4.0/}{\includegraphics[scale = 0.8]{img/by-nc-sa}}
\end{center}
\fi

Il progetto è ospitato su GitHub, dove si può trovare la versione più recente, al link \url{https://github.com/SciSNS-2017/SD-book}. Il progetto in cui sono raccolte le lezioni singolarmente a partire dalle quali è stato creato questo libro è sempre ospitato su GitHub, dove si può trovare la versione più recente, al link \url{https://github.com/SciSNS-2017/Sistemi-Dinamici}. Ogni suggerimento (errori, soluzioni di esercizi, \linebreak contributi vari, ecc.) è sempre gradito e può essere segnalato creando una \emph{issue} su GitHub oppure contattando direttamente gli autori.

\tableofcontents

\mainmatter
\chapter{Introduzione}
%
% \section{Lezione del 10/10/2018 [Marmi]
%

\section{Che cos'è un sistema dinamico?}
\begin{definition}[gruppo]
	Un gruppo è una coppia $ \mathcal{G} \coloneqq (G, \star) $ dove $ G $ è un insieme e $ {\star \colon G \times G \to G} $ è un'operazione binaria che gode delle seguenti proprietà
	\begin{enumerate}[label=(\roman*)]
		\item \emph{associativa}: $ \forall g_1, g_2, g_3 \in G, \ g_1 \star (g_2 \star g_3) = (g_1 \star g_2) \star g_3 $;
		\item \emph{elemento neutro sinistro}: $  \exists e \in G : \forall g \in G, \ g \star e = g $;
		\item \emph{inverso sinistro}: $ \forall g \in G, \exists g^{-1} \in G : g \star g^{-1} = e $.
	\end{enumerate}
	A partire da queste si mostra facilmente che l'elemento neutro destro è anche elemento neutro sinistro, l'inverso destro è anche inverso sinistro, che l'elemento neutro e l'inverso sono unici. \\
	Se non ci sono ambiguità circa l'operazione definita su $ G $ indicheremo più semplicemente il gruppo $ \mathcal{G} $ facendo riferimento al solo insieme $ G $. 
\end{definition}

\begin{definition}[sistema dinamico] \label{def:sistema-dinamico}
	Un sistema dinamico è una terna $ (\mathcal{G}, \mathcal{X}, \Phi) $ dove $ {\mathcal{G} \coloneqq (G, \star)} $ è un (semi-)gruppo\footnote{Per \emph{semigruppo} si intende una coppia $ (G,\star) $ dove $ \star $ è associativa.}, $ \mathcal{X} $ è uno spazio, cioè un insieme $ X $ dotato di una qualche struttura, e 
	\begin{align*}
		\Phi \colon G \times X & \to X \\
		(g, x) & \mapsto \Phi(g, x) = \Phi_g(x)
	\end{align*}
	è un'applicazione tale che
	\begin{enumerate}[label=(\roman*)]
		\item $ \forall x \in X, \ \Phi_e(x) = x $ dove $ e $ è l'elemento neutro di $ G $, cioè $ \Phi_e = \Id_X $;
		\item $ \forall g_1, g_2 \in G, \forall x \in X, \ \Phi_{(g_1 \star g_2)}(x) = (\Phi_{g1} \circ \Phi_{g_2})(x) $ cioè $ \Phi_{(g_1 \star g_2)} = \Phi_{g1} \circ \Phi_{g_2} $.
	\end{enumerate}
	Più brevemente diciamo che un sistema dinamico è l'\emph{azione} di un gruppo $ G $ su uno spazio $ X $ definita da una mappa $ \Phi $. 
\end{definition}

Nella maggior parte dei casi useremo come gruppo insiemi numerici $ \N $, $ \Z $ e $ \R $ con le usuali operazioni. Nei primi due casi parleremo di sistemi a \emph{tempo discreto} mentre nell'ultimo di sistemi a \emph{tempo continuo}. Come spazio $ \mathcal{X} $ useremo spesso uno \emph{spazio metrico compatto} (e.g. la sfera $ \S^d $, il toro $ \T^d $ \footnote{$ \T^d \coloneqq \faktor{\R^d}{\Z^d} $.} o un intervallo chiuso $ [a, b] $), uno \emph{spazio di misura} o gli insiemi $ \R^d $ e $ \C $ con le usuali strutture. Se non ci sono ambiguità circa la struttura definita su $ X $ indicheremo più semplicemente lo spazio $ \mathcal{X} $ facendo rifermento al solo insieme $ X $. \\

Per quanto riguarda la mappa $ \Phi $ osserviamo che per definizione $ \Phi_g \in \End{(X)} $ ovvero è un \emph{endomorfismo} su $ X $. Tuttavia spesso penseremo a $ \Phi_g \in \Aut{(X)} $ ovvero un \emph{automorfismo} cioè un endomorfismo invertibile. \\

Una tipo di sistema dinamico a tempo discreto di uso frequente è l'iterazione di una mappa da $ X $ in sé. Data $ f \in \End{(X)} $, per ogni $ n \in \N $ poniamo $ f^n \coloneqq f \circ \cdots \circ f $ ($ f $ composta $ n $ volte) con la convenzione che $ f^1 = f $ e $ f^0 = \Id_X $. Se consideriamo $ \N $ con l'operazione di addizione, l'applicazione $ \Phi^f $ data da $ \Phi_n^f(x) \coloneqq f^n(x) $ definisce un sistema dinamico. \\
Se prendiamo $ f \in \Aut{(X)} $ possiamo considerare la stessa costruzione usando come gruppo $ \Z $ e definendo $ f^{-n} $ come l'inversa di $ f^n $. \\
Nel seguito quando diremo che $ f \colon X \to X $ è un sistema dinamico sottintenderemo la costruzione appena data nell'esempio seguente a meno di ulteriori precisazioni. 


\begin{example}
	Partendo dalla costruzione appena data possiamo prendere $ X = [0, 1] $ e per $ \alpha \in \R $ la funzione $ f(x) \coloneqq x + \alpha \pmod{1} $. Osserviamo che essendo $ f $ invertibile possiamo definire come sopra l'applicazione $ \Phi $ su $ \Z $. Il sistema così definito è un prototipo di \emph{sistema periodico} se $ \alpha \in \Q $ e di \emph{sistema quasi-periodico} se $ \alpha \notin \Q $. 
	\iffigureon
	\begin{figure}[h!]
		\centering
		\input{img/sd1.tikz}
		\caption{\textcolor{red}{Completare questo schifo.}}
	\end{figure}
	\fi
\end{example}

Prendiamo come gruppo $ \R $ o $ [0, +\infty) $. In tale caso data l'applicazione $ \Phi_t(x) = \Phi(t, x) $ prende il nome di \emph{flusso} o \emph{semi-flusso} rispettivamente. \\

Un esempio di sistema dinamico a tempo continuo è dato da un'equazione differenziale ordinaria (ODE) del primo ordine\footnote{Di seguito considereremo quasi solo ODE del primo ordine in quanto equazioni differenziali di ordine superiore possono essere ricondotte a questa con il solito cambio di variabile a sistemi di ODE del primo ordine.} autonoma
\begin{equation} \label{eqn:ode-I-ordine}
	\begin{cases}
		\dot{x} = v(x) \\
		x(0) = x_0
	\end{cases}
\end{equation} 
dove $ x, x_0 \in \R^n $ e $ v \colon \R^n \to \R^n $ è un campo vettoriale. Se supponiamo che $ v $ sia di classe $ \mathcal{C}^1 $ allora abbiamo esistenza e unicità della soluzione \textcolor{red}{(e dipendenza continua dai parametri iniziali ??)}, cioè esiste $ \tau > 0 $ e un'unica funzione $ \phi \colon [0, \tau) \to \R^n $ tale che $ \phi(0) = x_0 $ e $ \phi'(t) = v(\phi(0)) $ per ogni $ t \in [0, \tau) $. \\
Se supponiamo per esempio che $ v $ sia un'applicazione lineare $ v(x) \coloneqq A x $ con $ A \in \mathrm{Mat}_{n \times n}(\R) $ allora abbiamo che la soluzione è prolungabile a tutto l'asse reale e introducendo la nozione di esponenziale di una matrice\footnote{%
	Data $ A \in \mathrm{Mat}_{n \times n}(\R) $ si pone 
	\[ \exp(A) \coloneqq \sum_{k = 0}^{+\infty} \frac{A^k}{k!}. \] 
}
si può scrivere nella forma 
\[ \phi(t) = \exp{\left(t \, A\right)} \, x_0. \]

\begin{definition}[Orbita e spazio delle orbite]
	Data $ f \in \Aut{(X)} $ e $ \Phi^f \colon \Z \times X \to X $ definiamo orbita di $ x \in X $ come 
	\[
	\mathcal{O}^f(x) \coloneqq \{f^n(x) : n \in \Z\}.
	\]
	Le orbite definiscono una naturale relazione di equivalenza $ x \sim y \iff \exists n \in \Z : y = f^n(x) \iff y \in \mathcal{O}^f(x) \iff x \in \mathcal{O}^f(y) $. Chiamiamo lo spazio quoziente $ \faktor{X}{\sim} $ spazio delle orbite. \textcolor{red}{Topologia quoziente?}
\end{definition}

\begin{definition}[orbita pre-periodica e periodica] 
    Sia $f\colon X \to X$ un sistema dinamico. Un'orbita $\mathcal{O}^f(x)$ si dice pre-periodica se contiene un numero finito di elementi. Se inoltre $ f $ è invertibile l'orbita si dice periodica e la sua cardinalità si dice periodo. \\
    Infine, se $f$ non è invertibile, possono esistere punti $ x $ (che costituiscono il pre-periodo) tali che $ \forall n > 0\ f^n (x) \neq x $.
\end{definition}

\begin{example}[Congettura di Collatz]
    Si consideri il sistema dinamico $f \colon \N \to \N$:
    \[
        f(n) \coloneqq
        \begin{cases}
            n/2  & \text{se $ n $ è pari}    \\
            3n+1 & \text{se $ n $ è dispari}
        \end{cases}
    \]
    La congettura\footnote{È attualmente un problema aperto.} di Collatz asserisce che tutti gli $n \in \N$ sono pre-periodici e che l'unico ciclo è $ 1 \to 4 \to 2 \to 1 $.
\end{example}

\section{Flussi vs tempo discreto: il pendolo semplice e la mappa standard}
Solitamente siamo interessati al comportamento asintotico dell'azione di $ \Phi $ e pertanto più essere utile rendere discreto un flusso continuo. Dato un flusso $ \Phi_t $ con $ t \in \R $ possiamo considerare la funzione $ f = \Phi_{\tau} $ con $ \tau > 0 $ e la mappa $ \Phi_n^f \coloneqq f^{n/\tau} $ con $ n \in \tau \Z $ che è un sistema dinamico a tempo discreto. Tuttavia bisogna osservare che la \emph{discretizzazione} dell'equazione differenziale che definisce un sistema dinamico continuo non è un mero artificio algebrico e il sistema a tempo discreto ottenuto può differire in modo significativo dal \emph{fotografare} un sistema continuo a tempi discreti. 

\begin{example}[pendolo semplice]
	Consideriamo il sistema dinamico descritto dall'equazione differenziale del pendolo semplice ($ g = 1 $)
	\[
		\ddot{x} = \sin{x}
	\]
	che può essere portato nella forma della \eqref{eqn:ode-I-ordine} ponendo $ y = \dot{x} $ da cui 
	\[
		\begin{cases}
			\dot{x} = y \\
			\dot{y} = \sin{x}
		\end{cases}
		\quad \rightarrow \quad
		\begin{pmatrix}
			\dot{x} \\ \dot{y}
		\end{pmatrix}
		=
		\begin{pmatrix}
			y \\ \sin{x}
		\end{pmatrix}
	\]
	dove $ v(x, y) = (y, \sin{x}) $. Nel linguaggio della Definizione \ref{def:sistema-dinamico}, $ G \coloneqq \R $, $ X \coloneqq \S^1 \times \R  $ e $ {\Phi(t, (x, y)) \coloneqq (\phi(t), \phi'(t))} $ dove $ \phi \colon \R \to \R $ è una soluzione dell'equazione differenziale. \\
	\textcolor{red}{$ E = \frac{1}{2}y^2 + \cos{x} $ integrale primo del moto e spazio  delle fasi?} \\
	A fissato \emph{step} $ \mu $ trasformiamo l'equazione differenziale del secondo ordine in \footnote{%
		\[
			\ddot{x}(t) \approx \frac{\dot{x}(t) - \dot{x}(t - \mu)}{\mu} \approx \frac{\frac{x(t + \mu) - x(t)}{\mu} - \frac{x(t) - x(t - \mu)}{\mu}}{\mu} = \frac{x(t + \mu) - 2x(t) + x(t - \mu)}{\mu^2}
		\]
	}
	\[
		x(t + \mu) - 2 x(t) + x(t - \mu) = \mu^2 \sin(x(t)).
	\]
	Posto ora $ \epsilon = \mu^2 $, $ x = x(t) $, $ x' = x(t + \mu) $, $ y = x(t) - x(t - \mu) $ e $ y' = x(t + \mu) - x(t) $ otteniamo il seguente sistema 
	\[
		\begin{cases}
			x' = x + y' \\
			y' = \epsilon \sin{x} + y 
		\end{cases}
	\]
	che è la \emph{mappa standard}. \\
	\textcolor{red}{Commenti...}
	
	\iffigureon
	\begin{figure}
		\centering
		\includegraphics{img/standard-map/fase-pendolo}
		\caption{Ritratto di fase del pendolo semplice}
		\label{fig:pendolo-fase}
	\end{figure}
	
	\begin{figure}
		\centering
		\subfloat{%
			\includegraphics[width=0.5\textwidth]{img/standard-map/sm0}%
		} 
		\subfloat{%
			\includegraphics[width=0.5\textwidth]{img/standard-map/sm1}%
		} \hfill
		\subfloat{%
			\includegraphics[width=0.5\textwidth]{img/standard-map/sm2}%
		} 
		\subfloat{%
			\includegraphics[width=0.5\textwidth]{img/standard-map/sm3}%
		} \hfill
		\subfloat{%
			\includegraphics[width=0.5\textwidth]{img/standard-map/sm4}%
		} 
		\subfloat{%
			\includegraphics[width=0.5\textwidth]{img/standard-map/sm5}%
		} \hfill
		\caption{Equazione del pendolo discretizzazione per diversi valori di $ \epsilon $}
		\label{fig:pendolo-ode-num}
	\end{figure}
	\fi
\end{example}

%
% \section{Lezione del 16/10/2018 [Marmi]}
%

\section{Coniugazione \textcolor{red}{e misure invarianti}}
\begin{definition}[Sistemi dinamici coniugati]
    Siano $f\colon X \to X$ e $g\colon Y \to Y$ due sistemi dinamici. Questi si dicono coniugati se esiste $h \colon X \to Y$ invertibile tale che $h \circ f = g \circ h$, cioè tale da far commutare il seguente diagramma:
    \begin{center}
        \begin{tikzcd}
            X \arrow[r, "f"] \arrow[d, "h"]  & X \arrow[d, "h"] \\
            Y \arrow[r, "g"] & Y
        \end{tikzcd}
    \end{center}
    Se $h$ è solamente surgettiva si dice che $g$ è un \emph{fattore} di $f$ oppure che $f$ è un'\emph{estensione} di $g$. Se invece $h$ è solo iniettiva allora si dice che $f$ è un \emph{sottosistema} di $g$.
\end{definition}

\begin{example} \label{ex:Ulam_Mandelbrot}
    Si considerino i seguenti sistemi dinamici $ \C \to \C$:
    \[ Q_\lambda(z) \coloneqq \lambda z (1-z) \]
    \[ P_c(z) \coloneqq z^2 + c \qquad \text{con } c = - \frac{\lambda^2}{4} +  \frac{\lambda}{2}. \]
    Le funzioni $ Q_\lambda $ sono dette \emph{trasformazioni di Ulam-Von Neumann}, mentre $P_c$ è la funzione che genera l'\emph{insieme di Mandelbrot}.
    I due sistemi risultano coniugati attraverso la funzione
    \[ h_\lambda(z) = -\lambda z + \frac{\lambda}{2}\;. \]
    Infatti si verifica che $ h\circ Q_\lambda = P_c \circ h $.
\end{example}

\begin{example} \label{ex:Ulam_tenda}
    Sia $ Q_4 \colon [0,1] \to [0,1] $ come definita nell'esempio \ref{ex:Ulam_Mandelbrot} e sia $ T\colon [0,1] \to [0,1] $ la mappa a tenda:
    \[
    T(x) \coloneqq
    \begin{cases}
    2x   & \text{se } 0 \leq x \leq 1/2 \\
    2-2x & \text{se } 1/2 \leq x \leq 1
    \end{cases}.
    \]
    Allora i due sistemi sono coniugati tramite $ h(x) = \sin^2\left(\frac{\pi x}{2}\right) $, cioè si ha $ Q_4\circ h = h\circ T $.
    Inoltre, poiché $ T $ conserva la misura di Lebesgue, usando la \eqref{eq:pushforward-misure} si ottiene che $ Q_4 $ conserva la misura:
    \[ \dif{h_\sharp \lambda}(x) = \frac{\dif x}{\pi\sqrt{x(1-x)}}\; . \]
    
    \iffigureon
    \begin{figure}
        \begin{center}
            \subfloat[Mappa a tenda]
            { \input{img/ulam-tenda/left.tikz} }
            \subfloat[Trasformazione $ Q_4 $]
            { \input{img/ulam-tenda/right.tikz} }
        \end{center}
        \caption{funzione dell'Esempio \ref{ex:Ulam_tenda}.}
    \end{figure}
    \fi
\end{example}

\begin{example}
    Sia $ Q_4\colon (0,1)\to(0,1) $, $ S\colon \R \to \R $ definita come
    \[ S(y) \coloneqq \log\left(\frac{4 e^y}{(1-e^y)^2}\right) \]
    e $ h\colon (0,1)\to\R $:
    \[ h(x) \coloneqq \operatorname{logit}(x) \coloneqq \log\left(\frac{x}{1-x}\right) \]
    Allora $ h\circ Q_4 = S \circ h $ e S conserva la misura:
    \[ \dif\mu(y) = \frac{\dif y}{\pi\left( e^{y/2} - e^{-y/2} \right) } \; . \]
\end{example}



\chapter{Dinamica Topologica}
\input{chapters/din-top}

\chapter{Dinamica Misurabile}
%
% \section{Lezione del 14/11/2018 [Marmi]}
%

\section{Sistemi dinamici su spazi di \textcolor{red}{misura/probabilità}}
\begin{definition}[sistema dinamico misurabile]
    Un sistema dinamico misurabile è una quaterna $ (X, \mathcal{A}, \mu, f) $ dove
    \begin{enumerate}[label=(\roman*)]
        \item $ (X, \mathcal{A}, \mu) $ è uno spazio di probabilità, cioè un insieme $ X $ (spazio delle fasi) con una $ \sigma $-algebra $ \mathcal{A} $ e una misura $ \mu $ tale che $ \mu(X) = 1 $;
        \item $ f \colon X \to X $ è una funzione misurabile e tale che $ \mu $ sia $ f $-invariante, cioè tale che $ \forall A \in \mathcal{A}, \ f_{\sharp}\mu (A) = \mu(f^{-1}(A)) = \mu(A) $;
        \item la dinamica è data dall'iterazione di $ f $.
    \end{enumerate}
\end{definition}

\textcolor{red}{Ho riscritto un po' la definizione seguente per renderla più simmetrica, controllare che sia equivalente a quella data a lezione}

\begin{definition}[isomorfismo di sistemi dinamici misurabili]
    Due sistemi dinamici misurabili $ (X, \mathcal{A}, \mu, f) $ e $ (Y, \mathcal{B}, \nu, g) $ si dicono isomorfi se esistono una funzione $ h \colon X \to Y $ e una funzione $ k \colon Y \to X $ che siano una l'inversa dell'altra a meno di un insieme di misura nulla e facciano commutare il seguente diagramma:
    \begin{center}
        \begin{tikzcd}
        X \arrow[r, "f"] \arrow[d, shift right, "h" left] & X \arrow[d, shift left, "h" right]\\
        Y \arrow[r, "g" below] \arrow[u, shift right, "k" right] & Y \arrow[u, shift left, "k" left]
        \end{tikzcd}
    \end{center}
    Più precisamente chiediamo che $ \exists X' \subseteq X : \mu(X \setminus X') = 0 $, $ \exists Y' \subseteq Y : \nu(Y \setminus Y') = 0 $ e $ \exists h \colon X' \to Y' $ e $ k \colon Y' \to X' $ tali che
    \begin{enumerate}[label=(\roman*)]
        \item $ h \circ k = \Id_{Y'} $;
        \item $ k \circ h = \Id_{X'} $;
        \item $ h $ faccia commutare il diagramma di sopra, cioè $ g \circ h = h \circ f $ $ \mu $-q.o. oppure $ k $ faccia commutare il diagramma, cioè $ f \circ k = k \circ g $ $ \nu $-q.o.;
        \item $ h $ sia misurabile, cioè $ \forall B \in \mathcal{B}, \ h^{-1}(B) \in \mathcal{A} $;
        \item $ \nu $ sia la misura immagine di $ \mu $ secondo $ h $, cioè $ {\forall B \in \mathcal{B}, \ h_\sharp \mu(B) = \mu(h^{-1}(B)) = \nu(B)} $;
        \item $ k $ sia misurabile, cioè $ \forall A \in \mathcal{A}, \ h^{-1}(A) \in \mathcal{B} $;
        \item $ \mu $ sia la misura immagine di $ \nu $ secondo $ k $, cioè $ \forall A \in \mathcal{A}, \ k_\sharp \nu(A) = \nu(k^{-1}(A)) = \mu(A) $.
    \end{enumerate}
\end{definition}

\begin{oss}
    Le nozioni di dinamica misurabile e teoria ergodica che andremo ad enunciare sono invarianti per isomorfismo di sistemi dinamici.
\end{oss}

In Tabella \ref{tab:ergodica-vs-probabilia}, si riporta un confronto tra il linguaggio probabilistico e quello usato in teoria ergodica.

\begin{table}[h!]
    \centering
    \begin{tabularx}{\textwidth}{cXX}
        & \textsc{Teoria ergodica} & \textsc{Probabilità} \\ \toprule
        $ X $ & spazio delle fasi & spazio dei campioni \\
        $ \mathcal{A} $ & $ \sigma $-algebra dei misurabili & collezione degli eventi \\
        $ \mu $ & misura $ \mu(A) $ & probabilità $ \PP{(x \in A)} $ \\
        $ \varphi \colon X \to \R $ & osservabile & variabile aleatoria \\
        $ \varphi_n \coloneqq (\varphi \circ f^{n})_{n \in \N} $ & valore di un'osservabile lungo un'orbita & processo stocastico con distribuzione  $ \PP{(\varphi_1 \in A_1, \ldots, \varphi_k \in A_k)} =  \mu\left(\bigcap_{j = 1}^{k} \{x \in X : \varphi(f^{j}(x)) \in A_j\}\right) $ \\
        & $ f $-invarianza di $ \mu $ & processo stazionario \\ \bottomrule
    \end{tabularx}
    \caption{Teoria ergodica e probabilità}
    \label{tab:ergodica-vs-probabilia}
\end{table}

\begin{exercise}
    Sia $ f \colon [0, 1] \to [0, 1] $ monotona e $ C^1 $ a tratti (anche numerabili). Su $ [0, 1] $ è posta una misura con densità $ \rho(x) $, cioè tale che $ \mu(A) \coloneqq \int_A \rho(x) \dif{x} $ dove $ \dif{x} $ è la misura di Lebesgue. Mostrare che $ \mu $ è $ f $-invariante se e solo se
    \[
        \sum_{x \in f^{-1}(\{y\})} \frac{\rho(x)}{\abs{f'(x)}} = \rho(y).
    \]
\end{exercise}

\begin{example}
    I sistemi dinamici nell'esempio \ref{ex:Ulam_tenda} sono isomorfi tramite la mappa $ h $ ivi definita.
\end{example}

\begin{exercise}[mappa di Gauss]
    Sia $ G\colon [0,1]\to [0,1] $ definita come $ G(x) \coloneqq \left\{ \frac{1}{x} \right\} $. Verificare che $ G $ conserva la misura con densità:
    \[ \dif{\mu}(x) = \frac{\dif x}{(\log 2)(1+x)} \, . \]
\end{exercise}

\begin{exercise}[mappa di Farey]
    Sia $ F\colon (0,1)\to (0,1) $ la mappa
    \[
        F(x) \coloneqq
        \begin{cases}
            \frac{x}{1-x}   & \text{se } 0 < x \leq \frac{1}{2} \\
            \frac{1}{x} - 1 & \text{se } \frac{1}{2} < x < 1
        \end{cases}
    \]
    Essa conserva la misura infinita $ \dif{\mu}(x) = \frac{\dif{x}}{x} $.
\end{exercise}

\begin{exercise}
    Sia $ f\colon \R\to\R $ definita come $ f(x) \coloneqq \frac{1}{2} \left( x - \frac{1}{x} \right) $. Questa conserva la misura $ \dif{\mu}(x) = \frac{\dif{x}}{\pi(1+x^2)} $.
\end{exercise}

Il seguente teorema definisce un collegamento tra un sistema dinamico topologico e un sistema dinamico misurabile e stabilisce che il sistema dinamico dato dall'iterazione di $ f $ su uno spazio metrico compatto ammette almeno una misura invariante. 

\begin{thm}[Krylov–Bogolyubov]
    Sia $ (X, d, f) $ un sistema dinamico topologico. Allora esiste almeno una misura $ \mu $ di probabilità sui boreliani di $ X $ che sia $ f $-invariante.
\end{thm}

\textcolor{red}{Qualche commento su Poincaré...}

\begin{thm}[di ricorrenza di Poincaré]
    Sia $ (X, \mathcal{A}, \mu, f) $ un sistema dinamico misurabile. Allora $ \forall A \in \mathcal{A} $, per $ \mu $-q.o. $ x \in A $, $ x $ è ricorrente in $ A $. 
\end{thm}
\begin{proof}
    Sia $ A_r \coloneqq \{x \in A : x \text{ è ricorrente in } A\} \subseteq A $. Osserviamo che possiamo scrivere
    \[
    A_r = A \setminus \bigcup_{n \geq 1} B_n
    \]
    dove $ B_n $ è l'insieme degli $ x \in A $ che non visitano più $ A $ dopo il tempo $ n $ o più formalmente  
    \[
    B_n \coloneqq A \setminus \bigcup_{j \geq n} f^{-j}(A).
    \]
    Essendo $ A_r $ differenza e unione numerabile di insiemi misurabili, anche $ A_r $ misurabile. Osservando che $ A \subseteq \bigcup_{j \geq 0} f^{-j}(A) $ essendo $ f^{0}(A) = A $ otteniamo che 
    \[
    B_n \subseteq \bigcup_{j \geq 0} f^{-j}(A) \setminus  \bigcup_{j \geq n} f^{-j}(A).
    \]
    Definendo allora $ \overline{A} \coloneqq \bigcup_{j \geq 0} f^{-j}(A) $ abbiamo che 
    \[
    \mu(B_n) \leq \mu(\overline{A}) - \mu\left(\textstyle{\bigcup_{j \geq n}} f^{-j}(A)\right) = \mu(\overline{A}) - \mu\left(f^{-n}\left(\textstyle{\bigcup_{j \geq 0}} f^{-j}(A)\right)\right) = \mu(\overline{A}) - \mu(f^{-n}(\overline{A})).
    \]
    Ma $ \mu $ è $ f $-invariate quindi
    \[
    \mu(B_n) \leq \mu(\overline{A}) - f_{\sharp}\mu(\overline{A}) = \mu(\overline{A}) - \mu(\overline{A}) = 0.
    \]
    \textcolor{red}{Scrivendo i $ B_n $ come unione disgiunta} concludiamo che 
    \[
    \mu(A_r) = \mu(A) - \mu\left(\textstyle{\bigcup_{n \geq 1}} B_n\right) = \mu(A)
    \]
    cioè $ \mu $-q.o. $ x \in A $ è ricorrente in $ A $.
\end{proof}

\section{Ergodicità}

\begin{definition}[frequenze di visita]
    Sia $ (X, \mathcal{A}, \mu, f) $ un sistema dinamico misurabile. Dato $ A \in \mathcal{A} $, $ x \in A $ e $ n \in \N $ definiamo la frequenza media delle visite ad $ A $ dell'orbita di $ x $ da $ 0 $ a $ n $ come
    \[
        \nu(x, A, n) \coloneqq \frac{1}{n} \sum_{j = 0}^{n-1} \chi_A(f^{j}(x)).
    \]
    Definiamo inoltre
    \begin{align*}
        \overline{\nu}(x, A) & \coloneqq \limsup_{n \to +\infty} \nu(x, A, n) \\
        \underline{\nu}(x, A) & \coloneqq \liminf_{n \to +\infty} \nu(x, A, n)
    \end{align*}
    Se $ \overline{\nu} = \underline{\nu} $ allora definiamo la frequenza media delle visite ad $ A $ dell'orbita di $ x $:
    \[
        \nu(x, A) \coloneqq \lim_{n \to +i\infty} \frac{1}{n} \sum_{j = 0}^{n-1} \chi_A(f^{j}(x)).
    \]
\end{definition}

\textcolor{red}{\emph{Merge}-are i due enunciati.} \\

Quest'ultima è una buona definizione in virtù del seguente

\begin{thm}[Birkhoff]\label{thm:Birkhoff}
    Sia $ (X,\mathcal{A},\mu,f) $ un sistema dinamico misurabile. Allora $ \forall A\in\mathcal{A} $ e per $ \mu $-q.o. $ x\in X $
    \[ \exists \lim_{n \to +\infty} \nu(x,A,n) = \nu(x,A) \, . \]
    Inoltre, $ \forall \varphi \in L^1(X,\mathcal{A},\mu) $ e per $ \mu $-q.o. $ x\in X $
    \[ \exists \lim_{n \to +\infty} \frac{1}{n} \sum_{j=0}^{n-1} \varphi\circ f^j(x) \eqqcolon \tilde{\varphi}(x) \, . \]
    La funzione $ \tilde{\varphi} $ viene detta \emph{media temporale} dell'osservabile $ \varphi $.
\end{thm}

\begin{thm}[Birkoff] \label{thm:Birkoff}
    Sia $ (X, \mathcal{A}, \mu, f) $ un sistema dinamico misurabile. Allora $ \forall A \in \mathcal{A} $, per $ \mu $-q.o. $ x \in X $, $ \overline{\nu}(x, A) = \underline{\nu}(x, A) $ cioè
    \[
    \exists \, \lim_{n \to +\infty} \frac{1}{n} \sum_{j = 0}^{n-1} \chi_A(f^{j}(x)) \eqqcolon \nu(x, A).
    \]
    Inoltre $ \forall \varphi \in L^1(X, \mathcal{A}, \mu; \R) $ e per $ \mu $-q.o. $ x \in X $
    \[
    \exists \, \lim_{n \to +\infty} \frac{1}{n} \sum_{j=0}^{n-1} (\varphi \circ f^j)(x) \eqqcolon \tilde{\varphi}(x).
    \]
\end{thm}
\begin{proof}
    content...
\end{proof}

\begin{definition}[sistema ergodico]
    Un sistema dinamico misurabile $ (X, \mathcal{A}, \mu, f) $ si dice ergodico se $ \forall A \in \mathcal{A} $ e per $ \mu $-q.o. $ x \in A $ vale
    \[
        \nu(x, A) = \mu(A)
    \]
    cioè se la frequenza statistica delle visite coincide con la probabilità a priori. 
\end{definition}

%
% \section{Lezione del 20/11/2018 [Marmi]}
%

% \emph{Setting}: $ (X, \mathcal{A}, \mu, f) $ è un sistema dinamico misurabile.

\begin{thm}
    Le seguenti proprietà sono equivalenti.
    \begin{enumerate}[label=(\roman*)]
        \item $ (X, \mathcal{A}, \mu, f) $ è ergodico.
        \item $ (X, \mathcal{A}, \mu, f) $ è \emph{metricamente indecomponibile} cioè $ \forall A \in \mathcal{A} : f(A) \subseteq A, \ \mu(A) \mu(X \setminus A) = 0 $, ovvero lo spazio delle fasi non può essere separato in due insiemi disgiunti $ f $-invarianti entrambi di misura non nulla.
        \item $ (X, \mathcal{A}, \mu, f) $ ha solo integrali primi del moto banali cioè $ \forall \varphi \in \textcolor{red}{L^1}(X, \mathcal{A}, \mu; \R) : \varphi \circ f = \varphi $ $ \mu $-q.o. si ha che $ \varphi $ è costante $ \mu $-q.o. in $ X $.
        \item $ \forall \varphi \in \textcolor{red}{L^1}(X, \mathcal{A}, \mu; \R) $ la media temporale di $ \varphi $ (che esiste per il Teorema \ref{thm:Birkhoff}) è $ \mu $-q.o. uguale alla media spaziale
        \[
        \lim_{n \to +\infty} \frac{1}{n} \sum_{j=0}^{n-1} (\varphi \circ f^j)(x) = \int_{X} \varphi \dif{\mu} \qquad \text{per } \mu\text{-q.o. } x \in X.
        \]
        \item $ \forall A, B \in \mathcal{A}, \ \displaystyle{\lim_{n \to +\infty} \frac{1}{n} \sum_{j=0}^{n-1} \mu(f^{-j}(A) \cap B) = \mu(A) \mu(B)} $ per $ \mu $-q.o. $ x \in X $, cioè il sistema dinamico è \emph{mescolante in media}.
    \end{enumerate}
\end{thm}
\begin{proof}
    Mostriamo le varie implicazioni.
    \begin{description}
        \item[$ (i) \Rightarrow (ii) $] Sia $ A\in\mathcal{A} : f(A) \subseteq A, \mu(A) > 0 $. Poiché $ A $ è $ f $-invariante, $ \forall x\in A\ \chi_A(f^j(x)) = 1 $ e dunque $ \nu(x,A) = 1 $ da cui, per l'ergodicità, $ \mu(A) = 1 $.
        \item[$ (ii) \Rightarrow (iii) $]
        \item[$ (iii) \Rightarrow (iv) $]
        \item[$ (iv) \Rightarrow (i) $]
        \item[$ (iv) \Rightarrow (v) $]
        \item[$ (v) \Rightarrow (ii) $]
    \end{description}
\end{proof}

\begin{proposition}\label{prop:rotazioni_erg}
    La rotazione $ R_\alpha \colon \T^1 \to \T^1 $ è ergodica se e solo se $ \alpha \in \R \setminus \Q $. 
\end{proposition}
\begin{proof}
    Usiamo la caratterizzazione con gli integrali primi. Prendo $ \varphi \colon \T^1 \to \R $ in \textcolor{red}{$ L^2 $} e lo sviluppo in serie di Fourier:
    \[
    \varphi(x) = \sum_{n \in \Z} \hat{\varphi}(n) e^{2\pi i n x}
    \]
    Componendola con la rotazione
    \[
    (\varphi \circ R_\alpha)(x) = \varphi(x + \alpha) = \sum_{n \in \Z} \hat{\varphi}(n) e^{2\pi i n \alpha} e^{2\pi i n x}.
    \]
    Affinché $ \varphi $ sia un integrale primo deve valere $ \hat{\varphi}(n) \left(e^{2\pi i n \alpha} - 1 \right) = 0. $ per ogni $ n \in \Z $.
    Ora se $ \alpha \in \R \setminus \Q, \ e^{2\pi i n \alpha} \neq 1 $ per ogni $ n \neq 0 $ così $ \varphi(x) = \hat{\varphi}(0) $ cioè $ \varphi $ è costante q.o. Per mostrare l'implicazione inversa supponiamo per assurdo che $ \alpha \in \Q $ e della forma $ p/q $; allora $ e^{2\pi i n \alpha} - 1 = 0 $ per ogni $ n $ della forma $ kq $ con $ k \in \Z $ da cui possiamo trovare un integrale primo non costante contro l'ipotesi che $ R_\alpha $ fosse ergodico.
\end{proof}

\begin{exercise}\label{ex:potenze_di_due_cancro}
    Sia $ x_j = \text{cifra più significativa di } 2^j $. Calcolare la frequenza di ciascuna cifra nella successione $ (x_j)_{j\in\N} $.
\end{exercise}
\begin{solution}
    Chiamiamo $ c_n $ la cifra più significativa di $ 2^n $, cioè il numero in $ \{1, \cdots, 9\} $ tale che
    \[ c_n \cdot 10^s \leq 2^n < (c_n+1) \cdot 10^s \]
    dove $ s = \lfloor \log_{10} 2^n \rfloor $. Prendendo ora il logaritmo si ha $ \log_{10}c_n + s \leq n\log_{10}2 < \log_{10}(c_n+1) + s $ e quindi
    \[ \log_{10}c_n \leq \{n\log_{10}2\} < \log_{10}(c_n+1) \]
    Considerando ora le rotazioni $ R_{\log_{10}(2)} \colon \T^1\to\T^1 $ possiamo riscrivere $ \{ n\log_{10}2 \} = R^n_{\log_{10}(2)}(0) $;
    se suddividiamo il toro come $ \T^1 = \sqcup_{k=1}^9 I_k $ con $ I_k = \left[\log_{10}k,\log_{10}(k+1)\right) $ la condizione che la cifra più significativa di $ 2^n $ sia $ c_n $ si traduce in
    \[ R^n_{\log_{10}(2)}(0) \in I_c \, . \]
    La sequenza delle potenze di 2 cercata è dunque la dinamica simbolica dell'orbita di 0 tramite la rotazione di $ \log_{10}2 $ secondo la suddetta partizione.
    
    Il sistema dinamico $ (\T^1, \mathcal{M}, \lambda, R_{\log_{10}(2)}) $ è ergodico per la proposizione \ref{prop:rotazioni_erg} in quanto $ \log_{10}2 $ è irrazionale; dunque la frequenza di visita dell'orbita di 0 agli intervalli $ I_k $ è uguale alla misura di Lebesgue degli intervalli stessi:
    \[ \nu(0,I_c) = \log_{10}\left(1+\frac{1}{c}\right) \, . \]
\end{solution}

\begin{example}[successione di Kolakoski]
    \textcolor{red}{mancante}
\end{example}

\begin{exercise}
    Mostrare che le dilatazioni sul toro sono ergodiche.
\end{exercise}
\begin{solution}
    Sia $ E_m\colon\T^1\to\T^1 $, $ E_m(x) = mx\pmod{1} $ per $ m\in\Z,\ \abs{m} \geq 2 $. Prendiamo $ \varphi\colon\T^1\to\R $ tale che $ \varphi\circ f = \varphi $; espandiamo $ \varphi $ in serie di Fourier e imponiamo che sia un integrale del moto
    \[ \varphi(x) = \sum_{n\in\Z} \hat\varphi(x) e^{2\pi i n x} = \sum_{n\in\Z} \hat\varphi(x) e^{2\pi i n m x} = \varphi(E_m(x)) \quad \forall x\in\T^1 \]
    Per l'ortogonalità della base di Fourier le due somme devono essere eguali termine a termine. Deve dunque valere che $ nx(1-m) = k $ per ogni $ k\in\Z $. L'equazione è banalmente verificata per $ n = 0 $. Se $ n\neq 0 $ basta prendere $ x\in\R\setminus\Q $ affinché l'equazione non sia verificata per nessun $ k\in\Z $.
\end{solution}

\begin{exercise}
    La trasformazione $ T_\alpha(x, y) = (x+\alpha, x+y) $ è ergodica se e solo se $ \alpha \in \R \setminus \Q $.
\end{exercise}

\section{Unica ergodicità}

\begin{proposition}
    Sia $ (X, \mathcal{A}, \mu_1, f) $ un sistema ergodico e $ \mu_2 $ una misura di probabilità su $ (X, \mathcal{A}) $, $ f $-invariante. Allora i seguenti fatti sono equivalenti:
    \begin{enumerate}[label=(\roman*)]
        \item $ \mu_1 \neq \mu_2 $;
        \item $ \mu_2 $ non è assolutamente continua rispetto a $ \mu_1 $, cioè $ \exists A \in \mathcal{A} $ tale che $ \mu_1(A) = 0 $ ma $ \mu_2(A) > 0 $;
        \item $ \exists A \in \mathcal{A} $ $ f $-invariante tale che $ \mu_1(A) = 0 $ e $ \mu_2(A) > 0 $.
    \end{enumerate}
\end{proposition}

Tale teorema stabilisce che un sistema dinamico misurabile può essere ergodico rispetto a due misure "che non si parlano". Ci sono tuttavia dei sistemi dinamici che ammettono una sola misura invariante. In tale caso si dà la seguente definizione.

\begin{definition}[sistema unicamente ergodico]
    $ (X; \mathcal{A}, \mu, f) $ si dice unicamente ergodico se esiste un'unica misura di probabilità su $ (X; \mathcal{A}) $ che sia $ f $-invariante.
\end{definition}

Tale definizione è ben posta perché un sistema unicamente ergodico è anche ergodico. Se infatti per assurdo $ (X, \mathcal{A}, \mu, f) $ non fosse ergodico esisterebbe $ A \in \mathcal{A} $ $ f $-invariante tale che $ \mu(A) > 0 $ e $ \mu(X \setminus A) > 0 $. Possiamo allora definire le misure 
\[
\nu_1(E) \coloneqq \frac{\mu(A \cap E)}{\mu(A)} 
\qquad 
\nu_2(E) \coloneqq \frac{\mu((X \setminus A) \cap E)}{\mu(X \setminus A)}
\] 
che sono misure di probabilità diverse e $ f $-invarianti contro l'ipotesi. Per l'invarianza basta osservare che per l'invarianza di $ \mu $ si ha
\[
\nu_1(f^{-1}(E)) = \frac{\mu(A \cap f^{-1}(E))}{\mu(A)} \textcolor{red}{=} \frac{\mu(f^{-1}(A) \cap f^{-1}(E))}{\mu(A)} = \frac{\mu(f^{-1}(A \cap E))}{\mu(A)} = \frac{\mu(A \cap E)}{\mu(A)} = \nu_1(E).
\] 

\begin{thm}
    Se $ (X, \mathcal{B}, \mu, f) $, con $ X $ spazio topologico e $ \mathcal{B} $ la $ \sigma $-algebra dei boreliani, è un sistema unicamente ergodico e $ \varphi \colon X \to \R $ è continua allora si ha convergenza uniforme della media temporale dell'osservabile:
    \[
    \frac{1}{n} \sum_{j=0}^{n-1} (\varphi \circ f^j)(x) \, \touf \, \int_X \varphi \dif{\mu}.
    \]
\end{thm}

\section{Iterated function systems \textcolor{red}{(mancante)}}

\section{Mescolamento di un sistema dinamico misurabile \textcolor{red}{(mancante)}}

\section{Schemi di Bernoulli \textcolor{red}{(mancante)}}



\chapter{Frazioni continue \textcolor{red}{(mancante)}}

\chapter{Entropia \textcolor{red}{(mancante)}}

\chapter{Catene di Markov \textcolor{red}{(mancante)}}

\chapter{Meccanica Statistica \textcolor{red}{(mancante)}}


\appendix
\chapter{Teoria della Misura}
\textcolor{red}{Da integrare con gli appunti di Grotto.}
\input{chapters/appendix/misura}

\chapter{Spazi di Hilbert}
%
% \section{Lezione del 23/10/18 [Bindini]}
%

\section{Prodotto scalare e struttura topologica}
\emph{Setting}: $ V $ è uno spazio vettoriale su $ \K = \R \text{ o } \C $. 

\begin{definition}[prodotto scalare]
    Una funzione $ {\langle \; , \, \rangle} \colon V \times V \to \K $ è un prodotto scalare se 
    \begin{enumerate}[label = (\roman*)]
        \item $ \forall v \in V, \ {\langle v, v \rangle} \geq 0 $ e $ {\langle v, v \rangle} \iff v = 0 $;
        \item $ \forall a_1, a_2 \in \K, \forall v_1, v_2, w \in V, \ {\langle a_1v_1 + a_2v_2, w \rangle} = a_1{\langle v_1, w\rangle} + a_2{\langle v_2, w\rangle} $;
        \item $ \forall b_1, b_2 \in \K, \forall w_1, w_2, v \in V, \ {\langle v, b_1w_1 + b_2w_2 \rangle} = \bar{b}_1{\langle v, w_1\rangle} + \bar{b}_2{\langle v, w_2\rangle} $
    \end{enumerate}
\end{definition}

Un prodotto scalare su $ V $ induce una funzione $ \norm{\;} \colon V \to \K  $ detta \emph{norma} definita come $ {\norm{v} \coloneqq \sqrt{{\langle v, v \rangle}}} $. Una norma a sua volta indice una distanza $ d \colon V \times V \to \K $ data da \linebreak $ d(v, w) \coloneqq \norm{v - w} $ che definisce una struttura di spazio metrico e di topologia.

\begin{proposition}[Cauchy-Schwarz]
    Per ogni $ v, w \in V $ vale $ \abs{{\langle v, w \rangle}} \leq \norm{v} \cdot \norm{w} $. 
\end{proposition}

\begin{definition}[sottospazio ortogonale]
    Se $ W \subseteq V $ è un sottospazio definiamo l'ortogonale di $ W $ come 
    \[
        W^{\perp} \coloneqq \{v \in V : \forall w \in W, \ {\langle v, w \rangle} = 0\}.
    \]
\end{definition}

\begin{lemma}
    Il prodotto scalare è una funzione continua rispetto alla topologia indotta dalla norma (e a quella euclidea in arrivo).
\end{lemma}
\begin{proof}
    Fissati $ (v_0, w_0) \in V \times V $, usando la triangolare e Cauchy-Schwarz
    \[
        \abs{{\langle v, w \rangle} - {\langle v_0, w_0 \rangle}} = \abs{{\langle v - v_0, w\rangle} - {\langle v_0, w - w_0 \rangle}} \leq \norm{v - v_0} \, \norm{w} + \norm{v_0} \, \norm{w - w_0}.
    \]
    Quindi se $ v \in B_\delta(v_0) $ e $ w \in B_\delta(w_0) $ si ha
    \[
        \abs{{\langle v, w \rangle} - {\langle v_0, w_0 \rangle}} \leq \delta(\norm{w} + \norm{w_0}) \leq \delta(\norm{w_0} + \norm{v_0} + \delta). \qedhere
    \]
\end{proof}

\begin{proposition}
    $ W^\perp $ è un sottospazio chiuso di $ V $.
\end{proposition}
\begin{proof}
    Il fatto che sia un sottospazio è ovvio per la linearità del prodotto scalare nella prima componente. Fissiamo ora $ w \in W $. Allora $ \{w\}^\perp = \{v \in V : {\langle v, w \rangle} = 0\} $ è un chiuso (controimmagine continua di un chiuso) da cui $ W^\perp = \bigcap_{w \in W} \{w\}^\perp $ è chiuso essendo intersezione di chiusi. 
\end{proof}

\begin{proposition}
    Vale che $ (W^\perp)^\perp = \overline{W} $.
\end{proposition}
\begin{proof}
    Se $ w \in W $ allora $ {\langle v, w \rangle} = 0 $ per ogni $ v \in W^\perp $ da cui $ W \subseteq (W^\perp)^\perp \Rightarrow \overline{W} \subseteq (W^\perp)^\perp $ essendo l'ortogonale di $ W^\perp $ un chiuso. \textcolor{red}{Viceversa}
\end{proof}

\section{Basi, completezza e spazi di Hilbert}

\begin{definition}[base ortonormale]
    Una base ortonormale di $ V $ è un insieme $ \{e_j\}_{j \in J} $ con $ J $ al più numerabile tale che
    \begin{enumerate}[label=(\roman*)]
        \item $ \forall j, k \in J, \ {\langle e_j, e_k \rangle} = \delta_{jk} $;
        \item\label{def:coin:base} $ \forall v \in V, \ \exists \{v_j\}_{j \in J} \subseteq \K : v = \displaystyle{\sum_{j \in J} v_j e_j} $. 
    \end{enumerate}
\end{definition} 

Nel caso in cui $ J $ sia numerabile la condizione \ref{def:coin:base} è da intendersi come convergenza rispetto alla norma, cioè
\[
    \norm{v - \sum_{j = 1}^{N} v_j e_j} \to 0 \quad \text{ per } N \to +\infty
\]

\begin{proposition}
   Uno spazio vettoriale $ V $ è separabile\footnote{Uno spazio topologico si dice separabile se contiene un sottoinsieme denso numerabile.} se e solo se ammette una base al più numerabile.
\end{proposition}

\begin{definition}[spazio di Hilbert]
    $ V $ si dice spazio di Hilbert se è separabile e completo rispetto alla norma indotta dal prodotto scalare.
\end{definition}

\begin{thm}[criterio di completezza]
    Sia $ (V, \norm{\;}) $ uno spazio normato. Allora le seguenti proprietà sono equivalenti.
    \begin{enumerate}[label=(\roman*)]
        \item $ V $ è completo.
        \item $ \forall (v_k)_{k \in \N} \subseteq V, \ \displaystyle{\sum_{k \in \N} \norm{v_k} < +\infty \Rightarrow \sum_{k \in \N} v_k} \text{ converge} $. 
    \end{enumerate}
\end{thm}

\begin{example}[$ \R^n, \C^n $]
    Lo spazio $ \C^n $ ($ \R^n $) con il prodotto hermitiano standard (scalare standard) $ {\langle v, w \rangle} = v \cdot w \coloneqq v_1\bar{w}_1 + \ldots v_n\bar{w}_n $ è uno spazio di Hilbert. Più in generale su $ \R^n $ possiamo definire il prodotto scalare $ {\langle v, w \rangle} = v \cdot Qw $ con $ Q $ matrice simmetrica definita positiva che dota $ \R^n $ di una struttura di spazio di Hilbert, di cui una base ortonormale è data dagli autovettori di $ Q $.  
\end{example}

\begin{example}[spazio $ L^2 $]
    Consideriamo lo spazio $ L^2(X, \mathcal{F}, \mu) \coloneqq \faktor{\mathscr{L}^2(X,\mathcal{F},\mu)}{\sim} $ dove
    \[
        \mathscr{L}^2(X,\mathcal{F},\mu) \coloneqq \left\{f \colon X \to \K \ \text{ misurabili tali che } \int_{X} \abs{f}^2 \dif{\mu} < +\infty\right\}
    \]
    e $ f \sim g \iff \mu\text{-q.o.} \ f = g $. Su $ L^2(X, \mathcal{F}, \mu) $ definiamo il prodotto scalare 
    \[
        {\langle f, g \rangle} \coloneqq \int_{X} f(x) \overline{g(x)} \dif{\mu(x)}
    \]
    che induce la norma
    \[
        \norm{f}_2 \coloneqq \left(\int_{X} \abs{f(x)}^2 \dif{\mu(x)}\right)^{1/2}.
    \]
    $ L^2(X, \mathcal{F}, \mu) $ con questo prodotto scalare è uno spazio di Hilbert. 
\end{example}

\begin{example}[spazio $ l^2 $]
    Consideriamo lo spazio
    \[
        \ell^2(\N) \coloneqq \left\{(x_n)_{n \in \N} \subseteq \K : \sum_{n \in \N} \abs{x_n}^2 < +\infty\right\}.
    \]
    Tale spazio può essere visto come un esempio di $ \mathscr{L}^2(X, \mathcal{F}, \mu) $ in cui $ X = \N $, $ \mathcal{F} = \mathscr{P}(\N) $ e $ \mu(E) = \card{E} $ (misura conta punti). Infatti in tale caso le funzioni $ f \colon \N \to \K $ sono successioni a valori nel campo e l'integrale si riduce a una sommatoria. Osserviamo inoltre che, poiché l'unico insieme di misura nulla è il vuoto, la relazione di equivalenza è banale, cioè le classi di equivalenza sono i singoletti di $ \ell^2(\N) $: $ l^2(\N) \simeq \ell^2(\N) = \mathscr{L}^2(\N, \mathscr{P}(\N), \mu) $. Il prodotto scalare su $ l^2(\N) $ è
    \[
        {\langle (x_n), (y_n) \rangle} = \sum_{n \in \N} x_n \bar{y}_n.
    \]
    Tale prodotto scalare è ben definito essendo la serie assolutamente convergente (infatti $ \abs{x_n \bar{y}_n} \leq \abs{x_n}^2 /2 + \abs{y_n}^2 /2 $). Una base di $ l^2(\N) $ è $ \{e^j\}_{j \in \N} $ dove $ e^j_n \coloneqq \delta_{jn} $.
\end{example}

\begin{proposition}
    Lo spazio $ l^2(\N) $ è completo. 
\end{proposition}
\begin{proof}
    \textcolor{red}{Pezzi di 2 dimostrazioni nel \texttt{tex}.}
    \iffalse
    Sia $ (x_k)_{k \in \N} \coloneqq (x_k(n))_{k \in \N} \subseteq l^2(\N) $ una successione di Cauchy. Osserviamo che per ogni $ n \in \N $ e per ogni $ k_1, k_2 \in \N $ vale
    \[
        \abs{x_{k_1}(n) - x_{k_2}(n)}^2 \leq \sum_{n \in \N} \abs{x_{k_1}(n) - x_{k_2}(n)}^2 = \norm{x_{k_1} - x_{k_2}}_2^2.
    \]
    da cui $ \abs{x_{k_1}(n) - x_{k_2}(n)} \leq \norm{x_{k_1} - x_{k_2}}_2 $. Quindi per ogni $ n \in \N $, $ (x_k(n))_{k \in \N} \subseteq \K $ è di Cauchy e pertanto converge. Sia $ \tilde{x}(n) $ il limite e $ \tilde{x} = (\tilde{x}(n))_{n \in \N} $. \\
    Per prima cosa mostriamo che $ \tilde{x} \in l^2(\N) $. Essendo la convergenza di $ (x_k(n))_{k \in \N} $ uniforme in $ n $ si ha
    \begin{align*}
        \norm{\tilde{x}}_2^2 & = \sum_{n \in \N} \abs{\tilde{x}(n)}^2 = \sum_{n \in \N} \abs{\lim_{k \to +\infty} x_k(n)}^2 \\
        & = \sum_{n \in \N} \lim_{k \to +\infty} \abs{x_k(n)}^2 = \lim_{k \to +\infty} \sum_{n \in \N} \abs{x_k(n)}^2 \\
        & = \lim_{k \to +\infty} \norm{x_k}^2_2 = \left(\lim_{k \to +\infty} \norm{x_k}_2\right)^2.
    \end{align*}
    Ma $ \abs{\norm{x_{k_1}}_2 - \norm{x_{k_2}}_2} \leq \norm{x_{k_1} - x_{k_2}} $ quindi $ (\norm{x_k}_2)_k \subseteq \K $ è di Cauchy e quindi converge. Essendo allora $ x_k \in l^2(\N) $ per ogni $ k $ concludiamo che anche il limite di $ (\norm{x_k}_2) $ è finito da cui $ \norm{\tilde{x}}_2 < +\infty $ cioè $ \tilde{x} \in l^2(\N) $. \\
    Mostriamo ora che $ (x_k) $ converge a $ \tilde{x} $ in $ l^2(\N) $ cioè che 
    \[
        \lim_{k \to +\infty} \norm{x_k - \tilde{x}}_2^2 = \lim_{k \to +\infty} \sum_{n \in \N} \abs{x_k(n) - \tilde{x}(n)}^2 = 0.
    \]
    Osserviamo che $  $ \\


    DIM ALTERNATIVA \\
    tale che $ \sum_{k \in \N} \norm{x_k}^2 < +\infty $. Osserviamo che
    \[
        \abs{x_k(n)}^2 \leq \sum_{n \in \N} \abs{x_k(n)}^2 = \norm{x_k}^2 \quad \Rightarrow \quad \abs{x_k(n)} \leq \norm{x_k}.
    \]
    Pertanto $ \bar{x}(n) \coloneqq \sum_{k \in \N} x_k(n) $ è convergente essendo assolutamente convergente ed è il candidato limite di $ \sum_{k \in \N} x_k(n) $. Vogliamo mostrare che $ \norm{x_k - \bar{x}}_2 \to 0 $ cioè che 
    \[
        \sum_{n \in \N} \; \abs{\bar{x}(n) - \sum_{k = 1}^{m} x_k(n)} \to 0 \quad \text{ per } m \to +\infty
    \]
    \fi
\end{proof}

\begin{exercise}
    Dimostrare che lo spazio $ L^2([0, 1]) $ è completo.
\end{exercise}

%
% \section{Lezione del 30/10/18 [Bindini]}
%

\section{Convergenza debole e insiemi completi}
\emph{Setting}: $ V $ spazio di Hilbert su $ \K = \R \text{ o } \C $.

\begin{definition}[convergenza debole]
    Diciamo che $ (v_n)_{n \in \N} \subseteq V $ converge debolmente a $ v \in V $, e scriviamo $ v_n \todeb v $, se $ \forall w \in V, \ {\langle v_n, w \rangle} \to {\langle v, w \rangle} $ (o analogamente $ {\langle v_n - v, w \rangle} \to 0 $) dove la convergenza del prodotto scalare è intesa rispetto alla distanza euclidea su $ \K $. 
\end{definition}

\begin{oss}
    Su $ \C^n $ o più in generale spazi di dimensione finita la convergenza debole è equivalente a quella rispetto alla norma. Infatti per avere la convergenza in norma è sufficiente guardare la convergenza sugli elementi di una base $ \{e^j\} $ e $ v_n^j = {\langle v_n, e^j\rangle} \to {\langle v, e^j\rangle} = v^j $. Il viceversa è invece ovvio.
\end{oss}

\begin{lemma}
    Sia $ (v_n)_{n \in \N} \subseteq V $. Se $ v_n \to v $ allora $ v_n \todeb v $.
\end{lemma}

\begin{lemma}
    Sia $ (v_n)_{n \in \N} \subseteq V $. Se $ v_n \todeb v $ e $ \norm{v_n} \to \norm{v} $ allora $ v_n \to v $.
\end{lemma}
\begin{proof}
    Infatti $ \norm{v_n - v}^2 = \norm{v_n}^2 + \norm{v}^2 - 2 {\langle v_n, v \rangle} \to 2\norm{v}^2 - 2\norm{v}^2 = 0 $.
\end{proof}

\begin{thm} \label{thm:Banach-Alaoglu-facile}
    La palla $ B_1(0) \subseteq V $ è relativamente compatta rispetto alla convergenza debole, cioè per ogni $ (v_n)_{n \in \N} \subseteq V $ con $ \norm{v_n} \leq 1 $ esiste una sottosuccessione $ (v_{n_k})_{k \in \N} \subseteq V $ e un $ v \in V $ tali che $ v_{n_k} \todeb v $.
\end{thm}
\begin{proof}
    (Idea)
    Per ogni $ v \in V $, consideriamo $ D_v \coloneqq \overline{B_{\norm{v}}(0)} \subseteq \C $ e l'insieme $ K \coloneqq \prod_{v \in V} D_v $. Essendo $ D_v $ compatto per ogni $ v $, $ K $ è compatto per il Teorema di Tychonoff. Sia ora 
    \begin{align*}
        \varphi \colon B_1(0) \subseteq V & \to K \\
        w & \mapsto ({\langle w, v \rangle})_{v \in V}
    \end{align*}
    Tale mappa è ben definita essendo $ \abs{{\langle v_n, v \rangle}} \leq \norm{w} \norm{v} \leq \norm{w} $. Basterà allora mostrare che la ``restrizione di $ \varphi $ nell'immagine'' $ \psi \colon B_1(0) \to \varphi(B_1(0)) $ è continua con inversa continua (prendendo su $ K $ al topologia indotta e su $ B_1(0) $ quella indotta della convergenza debole) e che $ \varphi(B_1(0)) $. Essendo allora $ \varphi(B_1(0)) $ chiusa in un compatto otteniamo che è compatta e dato che $ \psi $ è un omeomorfismo otteniamo che $ B_1(0) $ è compatta.
\end{proof}

\begin{proposition}
    Sia $ \{e_n\}_{n \in \N} $ un insieme ortonormale in $ V $. Le seguenti proprietà sono equivalenti. 
    \begin{enumerate}[label=(\roman*)]
        \item $ \forall v \in V, \ v = \sum_{n = 0}^{+\infty} {\langle v, e_n \rangle} e_n $;
        \item $ \forall v \in V, \ \exists \{a_n\}_{n \in \N} \subseteq \K : v = \sum_{n = 0}^{+\infty} a_n e_n $;
        \item $ \operatorname{Span}{(\{e_n\})} $\footnote{Con $ \operatorname{Span}{(\{w_n\})} $ si intende l'insieme delle combinazioni lineari \emph{finite} di elementi $ w_n $.} è denso in $ V $;
        \item $ {\langle v, e_n \rangle} = 0 \ \forall n \in \N \Rightarrow v = 0 $.
    \end{enumerate}
    Se è verificata una di queste proprietà si dice che $ \{e_n\} $ è un \emph{insieme completo}.
\end{proposition}
\begin{proof}
    Mostriamo le varie implicazioni.
    \begin{description}
        \item[$ (i) \Rightarrow (ii) $] Ovvio.
        \item[$ (ii) \Rightarrow (iii) $] Prendo $ v \in V $ che scrivo come $ \sum_{n = 0}^{+\infty} a_n v_n $. Ora se $ w_{N} \coloneqq \sum_{n = 0}^{N} a_ne_n $ ho per ipotesi che $ w_N \to v $. 
        \item[$ (iii) \Rightarrow (iv) $] Sia $ v \in V $ tale che $ \forall n \in \N, \ {\langle v, e_n \rangle} = 0 $. Siano $ (w_k)_{k \in \N} \subseteq \operatorname{Span}{(\{e_n\})} $ tali che $ w_k \to v $. Ma allora $ w_k \todeb v $ quindi $ {\langle v, v \rangle} = \lim_{k} {\langle v, w_k \rangle} = 0 $ essendo $ {\langle v, w_k \rangle} = \sum_{n = 0}^{N} a_n^k {\langle v, e_n \rangle} = 0 $. Così $ \norm{v} = 0 \Rightarrow v = 0 $.
        \item[$ (iv) \Rightarrow (i) $] Fissato $ v \in V $ poniamo $ v_N \coloneqq \sum_{n = 0}^{N} {\langle v, e_n \rangle} e_n $. Essendo $ \{e_n\} $ un insieme ortonormale
        \[
        {\langle v - v_N, v_N \rangle} = {\langle v, v_N \rangle} - {\langle v_N, v_N \rangle} = \sum_{n = 0}^{N} \overline{{\langle v, e_n \rangle}}{\langle v, e_n \rangle} - \norm{v_N}^2 = 0.
        \]
        Pertanto
        \[
        \norm{v}^2 = \norm{v - v_N}^2 + \norm{v_N}^2 \geq \norm{v_N}^2 = \sum_{n = 0}^{N} \abs{{\langle v, e_n \rangle}}^2.
        \]
        Ora $ \norm{v_N} \leq \norm{v} $ quindi per il Teorema \ref{thm:Banach-Alaoglu-facile} esiste $ (v_{N_k}) $ e $ \tilde{v} $ con $ \norm{\tilde{v}} \leq \norm{v} $ tale che $ v_{N_k} \todeb \tilde{v} $. Osservando quindi che
        \[
        {\langle v - \tilde{v}, e_n \rangle} = {\langle v, e_n \rangle} - {\langle \tilde{v}, e_n \rangle} = {\langle v, e_n \rangle} - \lim_{k \to + \infty} {\langle v_{N_k}, e_n \rangle} = {\langle v, e_n \rangle} - {\langle v, e_n \rangle} = 0.
        \]
        Per ipotesi allora $ \tilde{v} = v $. Mostriamo che $ (v_{N_k}) $ tende a $ v $ in norma. Infatti
        \[
        \norm{v - v_{N_k}}^2 = {\langle v - v_{N_k}, v - v_{N_k}\rangle} = {\langle v - v_{N_k}, v \rangle} - {\langle v - v_{N_k}, v_{N_k} \rangle} = {\langle v - v_{N_k}, v \rangle} \to 0.
        \]
        Finora abbiamo dunque estratto una sottosuccessione di $ v_N $ che tende a $ v $. Possiamo applicare lo stesso procedimento su un'arbitraria sottosuccessione di $ v_N $, estraendo da ciascuna una sotto-sottosuccessione convergente a $ v $. Per un noto lemma, quindi, l'intera successione $ v_N $ deve convergere a $ v $. \qedhere
    \end{description}
\end{proof}

\section{Spazio $ L^2([0, 2\pi]) $ e serie di Fourier}
Consideriamo lo spazio $ L^2([0, 2\pi]) $.
Una sua base ortonormale di è $ e_n(\theta) = \frac{1}{\sqrt{2\pi}}e^{i n \theta} $ con $ n \in \Z $ e $ \theta \in [0, 2\pi] $ ed è detta \emph{base di Fourier}.

\begin{oss}
    Al contrario delle funzioni di $ L^2(\T^1) $, le funzioni di $ L^2([0, 2\pi]) $ possono assumere valori diversi in $ 0 $ e $ 2\pi $. Prima di passare $ [0,2\pi] $ al quoziente, quindi, i due spazi non sono del tutto analoghi.
\end{oss}

\begin{definition}[polinomi trigonometrici]
    Chiamiamo polinomi trigonometrici di grado minore o uguale a $ N $ gli elementi di
    $ T_N \coloneqq \left\{\sum_{n = -N}^{N} a_n e^{i n \theta}, \ a_n \in \C\right\} $
\end{definition}

\begin{thm}
    La base di Fourier è un insieme completo di $ L^2([0,2\pi]) $.
\end{thm}
\begin{proof}
    Road map
    \begin{enumerate}
        \item Se $ f \in L^2([0, 2\pi]) $, $ \hat{f}(n) \coloneqq {\langle f, e_n \rangle} = \frac{1}{2\pi} \int_{0}^{2\pi} f(\theta) e^{-i n\theta} \dif{\theta} $ sono i candidati coefficienti;
        \item Mostro che per ogni $ p \in T_N $ vale $ \norm{f - \sum_{n = -N}^{N} \hat{f}(n) e_n}_2 \leq \norm{f - p}_2 $;
        \item Considero la famiglia di mollificatori $ \{Q_k\} $, con $ Q_k(t) \coloneqq \lambda_k (\frac{1 + \cos{t}}{2})^k $, dove $ \lambda_k $ è il fattore di normalizzazione. Questo è un polinomio trigonometrico di grado $ k $ e che ``tende alla delta di Dirac''. Mostro che
        \[
        (f * Q_k)(x) \coloneqq \int_{-\pi}^{\pi} f(y) Q_k(x - y) \dif{y} \overset{k}{\to} f(x)
        \]
        e uso il punto precedente per concludere. \textcolor{red}{Non chiaro}.
    \end{enumerate}
\end{proof}

\begin{exercise}
    Si calcolino i coefficienti di Fourier per la funzione
    \[
    f(\theta) \coloneqq
    \begin{cases}
    1  & \text{se } 0 \leq \theta \leq \pi \\
    -1 & \text{se } \pi < \theta \leq 2\pi
    \end{cases}
    \]
\end{exercise}

\textcolor{red}{Identità di Parseval e esempi}

\section{Convergenza della serie di Fourier  \textcolor{red}{(mancante)}}


\backmatter
\bibliography{bibliografia}
\bibliographystyle{alpha}
\nocite{stenberg}
\nocite{bs}
\nocite{ricci}
\nocite{khinchin}
\nocite{hw}
\nocite{shannon}
\nocite{norris}
\nocite{mm}

\end{document}
